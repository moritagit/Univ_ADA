\documentclass[class=jsarticle, crop=false, dvipdfmx, fleqn]{standalone}
%% preamble for Numerical-structure-analysis report

\input{/Users/User/Documents/Project/TeX/preamble/mypreamble}

%% titles
\title{先端データ解析論 レポート}
\author{37-196360 \quad 森田涼介}


%% setting for listings
\newtcbinputlisting[auto counter]{\reportlisting}[3][]{%
	listing file = {#3},
	listing options = {language=python, style=tcblatex, numbers=left, numberstyle=\tiny},
	listing only,
	breakable,
	toprule at break = 0mm,
	bottomrule at break = 0mm,
	left = 6mm,
	sharp corners,
	drop shadow,
	title = Listings \thetcbcounter : \texttt{#2},
	label = #1,
	}



%% title format
\usepackage{titlesec}
\titleformat{\section}{\LARGE}{宿題\thesection}{0zw}{}
\newcommand{\sectionbreak}{\clearpage}
\titleformat{\subsection}{\Large}{\Alph{subsection})}{0zw}{}

\begin{document}
\section{}

二乗ヒンジ損失に対する適応正則化分類の\(\bm{\mu}\)の解が次式で与えられることを示す。
\begin{equation}
    \hat{\bm{\mu}} = \tilde{\bm{\mu}} + \frac{y \max(0,\ 1 - \tilde{\bm{\mu}}^\mathrm{T} \phi(\bm{x}) y)}{\phi(\bm{x})^\mathrm{T} \tilde{\bm{\Sigma}} \phi(\bm{x}) + \gamma} \tilde{\bm{\Sigma}} \phi(\bm{x})
    \label{eq:objective}
\end{equation}

損失は,
\begin{align}
    J(\bm{\mu},\ \bm{\Sigma})
        & = \qty(\max(0,\ 1 - \bm{\mu}^\mathrm{T} \phi(\bm{x}) y))^2
            + \phi(\bm{x})^\mathrm{T} \bm{\Sigma} \phi(\bm{x})
            \notag \\
        & \ \qquad
            + \gamma \qty{
            \log\frac{\det(\tilde{\bm{\Sigma}})}{\det{\bm{\Sigma}}}
            + \mathrm{tr}\qty(\tilde{\bm{\Sigma}}^{-1} \bm{\Sigma})
            + (\bm{\mu} - \tilde{\bm{\mu}})^\mathrm{T} \tilde{\bm{\Sigma}}^{-1} (\bm{\mu} - \tilde{\bm{\mu}})
            - d
            }
    \label{eq:loss}
\end{align}
と表される。
\(f\)の\(x\)による劣微分を\(\partial_x f\)などと表すと,
この損失の\(\bm{\mu}\)による偏微分は,
\begin{equation}
    \pdv{J}{\bm{\mu}}
        = 2 \cdot \max(0,\ 1 - \bm{\mu}^\mathrm{T} \phi(\bm{x}) y) \cdot \partial_{\bm{\mu}} \max(0,\ 1 - \bm{\mu}^\mathrm{T} \phi(\bm{x}) y)
        + \gamma \qty(2 \tilde{\bm{\Sigma}}^{-1} \bm{\mu} - 2 \tilde{\bm{\Sigma}}^{-1} \tilde{\bm{\mu}})
    \label{eq:deriv_loss}
\end{equation}
となる。
\(z = \bm{\mu}^\mathrm{T} \phi(\bm{x}) y\)とおくと,
\begin{equation}
    \pdv{z}{\bm{\mu}} = y \phi(\bm{x})
    \label{eq:z}
\end{equation}
であり,
また,
\begin{align}
    & \max(0,\ 1 - z) =
        \begin{cases}
            1 - z & (z < 1) \\
            0 & (z \ge 1) \\
        \end{cases}
        \label{eq:max_z} \\
    & \partial_{\bm{\mu}} \max(0,\ 1 - z) =
        \begin{cases}
            - y \phi(\bm{x}) & (z < 1) \\
            \bm{0} & (z > 1) \\
            \qty[- y \phi(\bm{x}),\ \bm{0}] & (z = 1)
        \end{cases}
        \label{eq:subgrad_z} \\
\end{align}
である。

\(z > 1\)のとき,式(\ref{eq:deriv_loss}) -- (\ref{eq:subgrad_z})から,
\begin{equation}
    \pdv{J}{\bm{\mu}}
        = 2 \gamma \qty(\tilde{\bm{\Sigma}}^{-1} \bm{\mu} - \tilde{\bm{\Sigma}}^{-1} \tilde{\bm{\mu}})
        = \bm{0}
\end{equation}
よって,
\begin{equation}
    \hat{\bm{\mu}} = \tilde{\bm{\mu}}
    \label{eq:mu_hat_z_qt_1}
\end{equation}

\(z < 1\)のとき,式(\ref{eq:deriv_loss}) -- (\ref{eq:subgrad_z})から,
\begin{equation}
    \pdv{J}{\bm{\mu}}
        = 2 (1 - \bm{\mu}^\mathrm{T} \phi(\bm{x}) y) (- y \phi(\bm{x})) + 2 \gamma \qty(\tilde{\bm{\Sigma}}^{-1} \bm{\mu} - \tilde{\bm{\Sigma}}^{-1} \tilde{\bm{\mu}})
        = \bm{0}
\end{equation}
\(\bm{\mu}^\mathrm{T} \phi(\bm{x})\)はスカラーなので転置しても値は変わらないことから,
\begin{align}
    & (- y \phi(\bm{x})) (1 - y \phi(\bm{x})^\mathrm{T} \bm{\mu}) + \gamma \qty(\tilde{\bm{\Sigma}}^{-1} \bm{\mu} - \tilde{\bm{\Sigma}}^{-1} \tilde{\bm{\mu}})
    = \bm{0} \\
    & y^2 \phi(\bm{x}) \phi(\bm{x})^\mathrm{T} \bm{\mu} - y \phi(\bm{x}) + \gamma \tilde{\bm{\Sigma}}^{-1} \bm{\mu} - \gamma \tilde{\bm{\Sigma}}^{-1} \tilde{\bm{\mu}} = \bm{0} \\
    & \qty(\gamma \tilde{\bm{\Sigma}}^{-1} + y^2 \phi(\bm{x}) \phi(\bm{x})^\mathrm{T}) \bm{\mu} = \gamma \tilde{\bm{\Sigma}}^{-1} \tilde{\bm{\mu}} + y \phi(\bm{x}) \\
    & \qty(\tilde{\bm{\Sigma}}^{-1} + \frac{y^2}{\gamma} \phi(\bm{x}) \phi(\bm{x})^\mathrm{T}) \bm{\mu} = \tilde{\bm{\Sigma}}^{-1} \tilde{\bm{\mu}} + \frac{y}{\gamma} \phi(\bm{x})
\end{align}
いま,\(y \in \qty{-1,\ 1}\)を考えるから,\(y^2 = 1\)となる。
また,ShermanMorrison-Woodbury公式から,
\begin{equation}
    \qty(\tilde{\bm{\Sigma}}^{-1} + \frac{1}{\gamma} \phi(\bm{x}) \phi(\bm{x})^\mathrm{T})^{-1} = \qty(\tilde{\bm{\Sigma}} - \frac{\tilde{\bm{\Sigma}} \phi(\bm{x}) \phi(\bm{x})^\mathrm{T} \tilde{\bm{\Sigma}}}{\phi(\bm{x})^\mathrm{T} \tilde{\bm{\Sigma}} \phi(\bm{x}) + \gamma})
\end{equation}
が成立する。
よって,
\begin{align}
    \hat{\bm{\mu}}
        & = \qty(\tilde{\bm{\Sigma}}^{-1} + \frac{1}{\gamma} \phi(\bm{x}) \phi(\bm{x})^\mathrm{T})^{-1} \qty(\tilde{\bm{\Sigma}}^{-1} \tilde{\bm{\mu}} + \frac{y}{\gamma} \phi(\bm{x})) \\
        & = \qty(\tilde{\bm{\Sigma}} - \frac{\tilde{\bm{\Sigma}} \phi(\bm{x}) \phi(\bm{x})^\mathrm{T} \tilde{\bm{\Sigma}}}{\phi(\bm{x})^\mathrm{T} \tilde{\bm{\Sigma}} \phi(\bm{x}) + \gamma}) \qty(\tilde{\bm{\Sigma}}^{-1} \tilde{\bm{\mu}} + \frac{y}{\gamma} \phi(\bm{x})) \\
        & = \tilde{\bm{\mu}} - \frac{\tilde{\bm{\Sigma}} \phi(\bm{x}) \phi(\bm{x})^\mathrm{T}}{\phi(\bm{x})^\mathrm{T} \tilde{\bm{\Sigma}} \phi(\bm{x}) + \gamma} \tilde{\bm{\mu}} + \frac{y}{\gamma} \tilde{\bm{\Sigma}} \phi(\bm{x}) \qty(1 - \frac{\phi(\bm{x})^\mathrm{T} \tilde{\bm{\Sigma}}\phi(\bm{x})}{\phi(\bm{x})^\mathrm{T} \tilde{\bm{\Sigma}} \phi(\bm{x}) + \gamma}) \\
        & = \tilde{\bm{\mu}} - \frac{\tilde{\bm{\Sigma}} \phi(\bm{x}) \phi(\bm{x})^\mathrm{T}}{\phi(\bm{x})^\mathrm{T} \tilde{\bm{\Sigma}} \phi(\bm{x}) + \gamma} \tilde{\bm{\mu}} + \frac{y \tilde{\bm{\Sigma}} \phi(\bm{x})}{\phi(\bm{x})^\mathrm{T} \tilde{\bm{\Sigma}} \phi(\bm{x}) + \gamma} \\
        & = \tilde{\bm{\mu}} + \frac{\tilde{\bm{\Sigma}} \phi(\bm{x})}{\phi(\bm{x})^\mathrm{T} \tilde{\bm{\Sigma}} \phi(\bm{x}) + \gamma} \qty(y - \phi(\bm{x})^\mathrm{T} \tilde{\bm{\mu}})
\end{align}
\(y \in \qty{-1,\ 1}\)から,
\begin{align}
    \hat{\bm{\mu}}
        & = \tilde{\bm{\mu}} + \frac{y \tilde{\bm{\Sigma}} \phi(\bm{x})}{\phi(\bm{x})^\mathrm{T} \tilde{\bm{\Sigma}} \phi(\bm{x}) + \gamma} \qty(1 - \phi(\bm{x})^\mathrm{T} \tilde{\bm{\mu}} y) \\
        & = \tilde{\bm{\mu}} + \frac{y \qty(1 - \tilde{\bm{\mu}}^\mathrm{T} \phi(\bm{x}) y)}{\phi(\bm{x})^\mathrm{T} \tilde{\bm{\Sigma}} \phi(\bm{x}) + \gamma} \tilde{\bm{\Sigma}} \phi(\bm{x})
        \label{eq:mu_hat_z_lt_1}
\end{align}

\(z = 1\)のときの劣勾配を\(\bm{0}\)としてしまえば,
式(\ref{eq:mu_hat_z_qt_1}),(\ref{eq:mu_hat_z_lt_1})をまとめて,
\begin{equation}
    \hat{\bm{\mu}} = \tilde{\bm{\mu}} + \frac{y \max(0,\ 1 - \tilde{\bm{\mu}}^\mathrm{T} \phi(\bm{x}) y)}{\phi(\bm{x})^\mathrm{T} \tilde{\bm{\Sigma}} \phi(\bm{x}) + \gamma} \tilde{\bm{\Sigma}} \phi(\bm{x})
\end{equation}


\end{document}
