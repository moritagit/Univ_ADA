\documentclass[class=jsarticle, crop=false, dvipdfmx, fleqn]{standalone}
%% preamble for Numerical-structure-analysis report

\input{/Users/User/Documents/Project/TeX/preamble/mypreamble}

%% titles
\title{先端データ解析論 レポート}
\author{37-196360 \quad 森田涼介}


%% setting for listings
\newtcbinputlisting[auto counter]{\reportlisting}[3][]{%
	listing file = {#3},
	listing options = {language=python, style=tcblatex, numbers=left, numberstyle=\tiny},
	listing only,
	breakable,
	toprule at break = 0mm,
	bottomrule at break = 0mm,
	left = 6mm,
	sharp corners,
	drop shadow,
	title = Listings \thetcbcounter : \texttt{#2},
	label = #1,
	}



%% title format
\usepackage{titlesec}
\titleformat{\section}{\LARGE}{宿題\thesection}{0zw}{}
\newcommand{\sectionbreak}{\clearpage}
\titleformat{\subsection}{\Large}{\Alph{subsection})}{0zw}{}

\begin{document}
\section{}

標本\(\qty{(\bm{x}_i,\ y_i)}_{i=1}^{n}\)を考える。
\(\bm{x}\)は中心化されているとする。
つまり,
\begin{equation}
    \bm{\mu} = \frac{1}{n} \sum_{i=1}^{n} \bm{x}_i = \bm{0}
    \label{eq:centering}
\end{equation}
とする。
このとき,散布行列は,
\begin{equation}
    \bm{C} = \sum_{i=1}^{n} \bm{x}_i \bm{x}_i^\mathrm{T}
    \label{eq:scattering_matrix}
\end{equation}
クラス内散布行列は,
\begin{equation}
    \bm{S}^\mathrm{(w)} = \sum_{y=1}^{c} \sum_{i: y_i=y} \qty(\bm{x}_i - \bm{\mu}_y) \qty(\bm{x}_i - \bm{\mu}_y)^\mathrm{T}
    \label{eq:within_class_scattering_matrix}
\end{equation}
クラス間散布行列は,
\begin{equation}
    \bm{S}^\mathrm{(b)} = \sum_{y=1}^{c} n_y \bm{\mu}_y \bm{\mu}_y^\mathrm{T}
    \label{eq:between_class_scattering_matrix}
\end{equation}
である。
ここで,\(n_y\)はクラス\(y\)の標本数であり,
\begin{equation}
    \bm{\mu}_y = \frac{1}{n_y} \sum_{i: y_i=y} \bm{x}_i
    \label{eq:mean_y}
\end{equation}
である。

いま,散布行列について,
\begin{align*}
    \bm{C}
        = \sum_{i=1}^{n} \bm{x}_i \bm{x}_i^\mathrm{T}
        = \frac{1}{2} \qty(\sum_{i=1}^{n} \bm{x}_i \bm{x}_i^\mathrm{T} + \sum_{i'=1}^{n} \bm{x}_{i'} \bm{x}_{i'}^\mathrm{T})
\end{align*}
と変形できる。
\begin{equation*}
    \sum_{i=1}^{n} \bm{x}_i \bm{x}_i^\mathrm{T}
        = \sum_{i'=1}^{n} \frac{1}{n} \qty(\sum_{i=1}^{n} \bm{x}_i \bm{x}_i^\mathrm{T})
        = \frac{1}{n} \sum_{i, i'=1}^{n} \bm{x}_i \bm{x}_i^\mathrm{T}
\end{equation*}
であることから,
\begin{align*}
    \bm{C}
        = \frac{1}{2} \cdot \frac{1}{n} \sum_{i, i'}^{n} \qty(\bm{x}_i \bm{x}_{i}^\mathrm{T} + \bm{x}_{i'} \bm{x}_{i'}^\mathrm{T})
\end{align*}
となる。
ここで,式(\ref{eq:centering})から,
\begin{align*}
    \sum_{i, i'=1}^{n} \bm{x}_i \bm{x}_{i'}^\mathrm{T}
        & = \sum_{i=1}^{n} \bm{x}_i \qty(\sum_{i'=1}^{n} \bm{x}_{i'})^\mathrm{T} + \sum_{i'=1}^{n} \qty(\sum_{i=1}^{n} \bm{x}_i) \bm{x}_{i'}^\mathrm{T} \\
        & = 2 \sum_{i=1}^{n} \bm{x}_i \sum_{i'=1}^{n} \bm{x}_{i'}^\mathrm{T} \\
        & = 2 n\bm{\mu} \cdot n\bm{\mu}^\mathrm{T} \\
        & = \bm{0}
\end{align*}
が成立することを用いると,結局,
\begin{align}
    \bm{C}
        & = \frac{1}{2} \cdot \frac{1}{n} \sum_{i, i'=1}^{n} \qty(\bm{x}_i \bm{x}_{i}^\mathrm{T} - 2 \bm{x}_i \bm{x}_{i'}^\mathrm{T} + \bm{x}_{i'} \bm{x}_{i'}^\mathrm{T}) \notag \\
        & = \frac{1}{2} \sum_{i, i'=1}^{n} \frac{1}{n} \qty(\bm{x}_i - \bm{x}_{i'}) \qty(\bm{x}_i - \bm{x}_{i'})^\mathrm{T}
        \label{eq:C_pair}
\end{align}
得る。

同様に,クラス内散布行列は,
\begin{align*}
    \bm{S}^\mathrm{(w)}
        & = \sum_{y=1}^{c} \sum_{i: y_i=y} \qty(\bm{x}_i - \bm{\mu}_y) \qty(\bm{x}_i - \bm{\mu}_y)^\mathrm{T} \\
        & = \frac{1}{2} \sum_{y=1}^{c} \frac{1}{n_y} \sum_{i, i': y_{i, i'}=y} \qty{\qty(\bm{x}_i - \bm{\mu}_y)\qty(\bm{x}_i - \bm{\mu}_y)^\mathrm{T} + \qty(\bm{x}_{i'} - \bm{\mu}_y)\qty(\bm{x}_{i'} - \bm{\mu}_y)^\mathrm{T}}
\end{align*}
と変形でき,
ここで,式(\ref{eq:mean_y})から,
\begin{align*}
    \sum_{i, i': y_{i, i'}=y} \qty(\bm{x}_i - \bm{\mu}_y)\qty(\bm{x}_{i'} - \bm{\mu}_y)^\mathrm{T}
        & = 2 \sum_{i: y_i=y} \qty(\bm{x}_i - \bm{\mu}_y) \cdot \sum_{i': y_i'=y} \qty(\bm{x}_{i'} - \bm{\mu}_y)^\mathrm{T} \\
        & = 2 \qty(n_y \bm{\mu}_y - n_y \bm{\mu}_y) \qty(n_y \bm{\mu}_y - n_y \bm{\mu}_y)^\mathrm{T} \\
        & = \bm{0}
\end{align*}
となることから,結局,
\begin{align*}
    \bm{S}^\mathrm{(w)}
    & = \frac{1}{2} \sum_{y=1}^{c} \frac{1}{n_y} \sum_{i, i': y_{i, i'}=y} \qty{\qty(\bm{x}_i - \bm{\mu}_y)\qty(\bm{x}_i - \bm{\mu}_y)^\mathrm{T} - 2 \qty(\bm{x}_i - \bm{\mu}_y)\qty(\bm{x}_{i'} - \bm{\mu}_y)^\mathrm{T} + \qty(\bm{x}_{i'} - \bm{\mu}_y)\qty(\bm{x}_{i'} - \bm{\mu}_y)^\mathrm{T}} \\
    & = \frac{1}{2} \sum_{y=1}^{c} \frac{1}{n_y} \sum_{i, i': y_{i, i'}=y} \qty{\qty(\bm{x}_i - \bm{\mu}_y) - \qty(\bm{x}_{i'} - \bm{\mu}_y)} \qty{\qty(\bm{x}_i - \bm{\mu}_y) - \qty(\bm{x}_{i'} - \bm{\mu}_y)}^\mathrm{T} \\
    & = \frac{1}{2} \sum_{y=1}^{c} \frac{1}{n_y} \sum_{i, i': y_{i, i'}=y} \qty(\bm{x}_i - \bm{x}_{i'}) \qty(\bm{x}_i - \bm{x}_{i'})^\mathrm{T}
\end{align*}
となる。これは,
\begin{equation}
    Q_{i, i'}^\mathrm{(w)} =
        \begin{cases}
            1/n_y & (y_i = y_{i'} = y) \\
            0 & (y_i \neq y_{i'})
        \end{cases}
    \label{eq:Qw}
\end{equation}
なる\(Q_{i, i'}^\mathrm{(w)}\)を用いて,
\begin{equation}
    \bm{S}^\mathrm{(w)} = \frac{1}{2} \sum_{i, i'=1}^{n} Q_{i, i'}^\mathrm{(w)} \qty(\bm{x}_i - \bm{x}_{i'}) \qty(\bm{x}_i - \bm{x}_{i'})^\mathrm{T}
    \label{eq:Sw_pair}
\end{equation}
と表せる。

いま,\(\bm{S}^\mathrm{(w)}\)について変形すると,
\begin{align*}
    \bm{S}^\mathrm{(w)}
        & = \sum_{y=1}^{c} \sum_{i: y_i=y} \qty(\bm{x}_i - \bm{\mu}_y) \qty(\bm{x}_i - \bm{\mu}_y)^\mathrm{T} \\
        & = \sum_{y=1}^{c} \sum_{i: y_i=y} \qty(\bm{x}_i \bm{x}_i^\mathrm{T} - 2 \bm{\mu}_y \bm{x}_i^\mathrm{T} + \bm{\mu}_y \bm{\mu}_y^\mathrm{T}) \\
        & = \sum_{i=1}^{n} \bm{x}_i \bm{x}_i^\mathrm{T} - \sum_{y=1}^{c} 2 \bm{\mu}_y \sum_{i: y_i=y} \bm{x}_i^\mathrm{T} + \sum_{y=1}^{c} n_y \bm{\mu}_y \bm{\mu}_y^\mathrm{T} \\
        & = \sum_{i=1}^{n} \bm{x}_i \bm{x}_i^\mathrm{T} - \sum_{y=1}^{c} 2 \bm{\mu}_y n_y \bm{\mu}_y^\mathrm{T} + \sum_{y=1}^{c} n_y \bm{\mu}_y \bm{\mu}_y^\mathrm{T} \\
        & = \sum_{i=1}^{n} \bm{x}_i \bm{x}_i^\mathrm{T} - \sum_{y=1}^{c} n_y \bm{\mu}_y \bm{\mu}_y^\mathrm{T}
\end{align*}
となり,式(\ref{eq:scattering_matrix}),(\ref{eq:between_class_scattering_matrix})とから,
\begin{equation}
    \bm{C} = \bm{S}^\mathrm{(w)} + \bm{S}^\mathrm{(b)}
    \label{eq:relation}
\end{equation}
が成立する。

式(\ref{eq:C_pair}),(\ref{eq:Sw_pair}),(\ref{eq:relation})から,
\begin{align}
    \bm{S}^\mathrm{(b)}
        & = \bm{C} - \bm{S}^\mathrm{(w)} \notag \\
        & = \frac{1}{2} \sum_{i, i'=1}^{n} \qty(\frac{1}{n} - Q_{i, i'}^\mathrm{(w)}) \qty(\bm{x}_i - \bm{x}_{i'}) \qty(\bm{x}_i - \bm{x}_{i'})^\mathrm{T} \notag \\
        & = \frac{1}{2} \sum_{i, i'=1}^{n} Q_{i, i'}^\mathrm{(b)} \qty(\bm{x}_i - \bm{x}_{i'}) \qty(\bm{x}_i - \bm{x}_{i'})^\mathrm{T}
        \label{eq:Sb_pair} \\
    Q_{i, i'}^\mathrm{(b)}
        & = \frac{1}{n} - Q_{i, i'}^\mathrm{(w)} \notag \\
        & =
            \begin{cases}
                1/n - 1/n_y & (y_i = y_{i'} = y) \\
                1/n & (y_i \neq y_{i'})
            \end{cases}
        \label{eq:Qb}
\end{align}
となる。



\end{document}
