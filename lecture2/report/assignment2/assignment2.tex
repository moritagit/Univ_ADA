\documentclass[class=jsarticle, crop=false, dvipdfmx, fleqn]{standalone}
%% preamble for Numerical-structure-analysis report

\input{/Users/User/Documents/Project/TeX/preamble/mypreamble}

%% titles
\title{先端データ解析論 レポート}
\author{37-196360 \quad 森田涼介}


%% setting for listings
\newtcbinputlisting[auto counter]{\reportlisting}[3][]{%
	listing file = {#3},
	listing options = {language=python, style=tcblatex, numbers=left, numberstyle=\tiny},
	listing only,
	breakable,
	toprule at break = 0mm,
	bottomrule at break = 0mm,
	left = 6mm,
	sharp corners,
	drop shadow,
	title = Listings \thetcbcounter : \texttt{#2},
	label = #1,
	}



%% title format
\usepackage{titlesec}
\titleformat{\section}{\LARGE}{宿題\thesection}{0zw}{}
\newcommand{\sectionbreak}{\clearpage}
\titleformat{\subsection}{\Large}{\Alph{subsection})}{0zw}{}

\begin{document}
\section{}

線形モデル\(f_{\bm{\theta}} (\bm{x}) = \sum_{j=1}^b \theta_j \phi_j (\bm{x})\)を用いたL2正則化回帰に,
一つ抜き交差確認法を適用したときの二乗誤差\(L\)が次式のように表されることを示す。
\begin{align}
    & L = \frac{1}{n} ||\tilde{\bm{H}}^{-1} \bm{H} \bm{y}||^2 \\
    & \bm{H} = \bm{I} - \bm{\Phi} (\bm{\Phi}^\mathrm{T} \bm{\Phi} + \lambda \bm{I})^{-1} \bm{\Phi}^\mathrm{T} \\
    & \tilde{\bm{H}} : \text{\(\bm{H}\)と同じ対角成分を持ち,非対角成分は0}
\end{align}

訓練誤差\(L_\mathrm{train}\)は,
\begin{align}
    L_\mathrm{train} = ||\bm{\Phi \theta} - \bm{y} ||^2 + \frac{\lambda}{2} ||\bm{\theta}||^2
\end{align}
これを最小にする\(\bm{\theta}\)は,
上式の\(\bm{\theta}\)による偏微分が0である条件から得られて,
\begin{equation}
    \hat{\bm{\theta}} = (\bm{\Phi}^\mathrm{T} \bm{\Phi} + \lambda \bm{I})^{-1} \bm{\Phi}^\mathrm{T} \bm{y}
\end{equation}
となる。
これより,標本\((\bm{x}_i,\ y_i)\)を除いた標本群を用いて学習したときに得られるパラメータ\(\bm{\theta}\)は,
\begin{align}
    \hat{\bm{\theta}}_i
        & = (\bm{\Phi}_i^\mathrm{T} \bm{\Phi}_i + \lambda \bm{I})^{-1} \bm{\Phi}_i^\mathrm{T} \bm{y}_i \\
        & = (\bm{\Phi}^\mathrm{T} \bm{\Phi} + \lambda \bm{I} - \bm{\phi}_i \bm{\phi}_i^\mathrm{T})^{-1} (\bm{\Phi}^\mathrm{T} \bm{y} - \bm{\phi}_i y_i) \\
        & = (\bm{U} - \bm{\phi}_i \bm{\phi}_i^\mathrm{T})^{-1} (\bm{\Phi}^\mathrm{T} \bm{y} - \bm{\phi}_i y_i)
\end{align}
と表される。ここで,
\begin{equation}
    \bm{U} = \bm{\Phi}^\mathrm{T} \bm{\Phi} + \lambda \bm{I}
\end{equation}
とおいた。
また,ShermanMorrison-Woodbury公式を用いると,
\begin{equation}
    (\bm{U} - \bm{\phi}_i \bm{\phi}_i^\mathrm{T})^{-1}
    =
    \bm{U}^{-1} + \frac{\bm{U}^{-1} \bm{\phi}_i \bm{\phi}_i^\mathrm{T} \bm{U}^{-1}}{1 - \bm{\phi}_i^\mathrm{T} \bm{U}^{-1} \bm{\phi}_i}
\end{equation}
となることから,結局,
\begin{equation}
    \hat{\bm{\theta}}_i = \qty(\bm{U}^{-1} + \frac{\bm{U}^{-1} \bm{\phi}_i \bm{\phi}_i^\mathrm{T} \bm{U}^{-1}}{1 - \bm{\phi}_i^\mathrm{T} \bm{U}^{-1} \bm{\phi}_i}) (\bm{\Phi}^\mathrm{T} \bm{y} - \bm{\phi}_i y_i)
\end{equation}
これより,\(\alpha_i = \bm{\phi}_i^\mathrm{T} \bm{U}^{-1} \bm{\phi}_i\)とおくと,
\(y_i\)の予測値\(\hat{y}_i\)は,
\begin{align}
    \hat{y}_i
        & = \bm{\phi}_i^\mathrm{T} \hat{\bm{\theta}}_i \\
        & = \bm{\phi}_i^\mathrm{T} \qty(\bm{U}^{-1} + \frac{\bm{U}^{-1} \bm{\phi}_i \bm{\phi}_i^\mathrm{T} \bm{U}^{-1}}{1 - \alpha_i}) (\bm{\Phi}^\mathrm{T} \bm{y} - \bm{\phi}_i y_i) \\
        & = \frac{1}{1 - \alpha_i} \qty( \bm{\phi}_i^\mathrm{T} (1 - \alpha_i) \bm{U}^{-1} + \bm{\phi}_i^\mathrm{T} \bm{U}^{-1} \bm{\phi}_i \bm{\phi}_i^\mathrm{T} \bm{U}^{-1} ) (\bm{\Phi}^\mathrm{T} \bm{y} - \bm{\phi}_i y_i) \\
        & = \frac{1}{1 - \alpha_i} \qty( \bm{\phi}_i^\mathrm{T} \bm{U}^{-1} - \alpha_i \bm{\phi}_i^\mathrm{T} \bm{U}^{-1} + \alpha_i \bm{\phi}_i^\mathrm{T} \bm{U}^{-1} ) (\bm{\Phi}^\mathrm{T} \bm{y} - \bm{\phi}_i y_i) \\
        & = \frac{\bm{\phi}_i^\mathrm{T} \bm{U}^{-1}}{1 - \alpha_i} (\bm{\Phi}^\mathrm{T} \bm{y} - \bm{\phi}_i y_i) \\
        & = \frac{\bm{\phi}_i^\mathrm{T} \bm{U}^{-1} \bm{\Phi}^\mathrm{T} \bm{y} - \alpha_i y_i}{1 - \alpha_i}
\end{align}
よって,テスト誤差\(L\)は,
\begin{align}
    L
        & = \frac{1}{n} \sum_{i=1}^n (\hat{y}_i - y_i)^2 \\
        & = \frac{1}{n} \sum_{i=1}^n \qty(\frac{\bm{\phi}_i^\mathrm{T} \bm{U}^{-1} \bm{\Phi}^\mathrm{T} \bm{y} - \alpha_i y_i}{1 - \alpha_i} - y_i) \\
        & = \frac{1}{n} \sum_{i=1}^n \qty(\frac{\bm{\phi}_i^\mathrm{T} \bm{U}^{-1} \bm{\Phi}^\mathrm{T} \bm{y} - y_i}{1 - \alpha_i})
\end{align}
となる。

また,\(\bm{H}\)と\(\tilde{\bm{H}}\)について,
\begin{align}
    \bm{H}
        & = \bm{I} - \bm{\Phi} (\bm{\Phi}^\mathrm{T} \bm{\Phi} + \lambda \bm{I})^{-1} \bm{\Phi}^\mathrm{T} \\
        & = \bm{I} - \bm{\Phi} \bm{U}^{-1} \bm{\Phi}^\mathrm{T} \\
        & = \bm{I} -
            \begin{bmatrix}
                \bm{\phi}_1^\mathrm{T} \bm{U}^{-1} \bm{\phi}_1 & \cdots & \bm{\phi}_1^\mathrm{T} \bm{U}^{-1} \bm{\phi}_n \\
                \vdots & \ddots & \vdots \\
                \bm{\phi}_n^\mathrm{T} \bm{U}^{-1} \bm{\phi}_1 & \cdots & \bm{\phi}_n^\mathrm{T} \bm{U}^{-1} \bm{\phi}_n \\
            \end{bmatrix}
\end{align}
より,\(\bm{H}\)(及び\(\tilde{\bm{H}}\))の\(i\)番目対角成分は\(1 - \alpha_i\),
\(\tilde{\bm{H}}^{-1}\)の\(i\)番目対角成分は\(1/(1 - \alpha_i)\)となる。
また,
\begin{align}
    \bm{\phi}_i^\mathrm{T} \bm{U}^{-1} \bm{\Phi}^\mathrm{T} \bm{y} - y_i
        & = \sum_{j=1}^{n} \qty(\bm{\phi}_i^\mathrm{T} \bm{U}^{-1} \bm{\phi}_j y_j) - y_i \\
        & = -(\bm{Hy})_i
\end{align}
である。
以上より,テスト誤差は,
\begin{align}
    L
        & = \frac{1}{n} \sum_{i=1}^n \qty(\frac{\bm{\phi}_i^\mathrm{T} \bm{U}^{-1} \bm{\Phi}^\mathrm{T} \bm{y} - y_i}{1 - \alpha_i}) \\
        & = \frac{1}{n} \sum_{i=1}^n \qty(\frac{-(\bm{Hy})_i}{1 - \alpha_i})^2 \\
        & = \frac{1}{n} \sum_{i=1}^n \qty(\tilde{\bm{H}}_i^{-1} (\bm{Hy})_i)^2 \\
        & = \frac{1}{n} \sum_{i=1}^n \qty(\tilde{\bm{H}}^{-1} \bm{Hy})_i^2 \\
        & = \frac{1}{n} ||\tilde{\bm{H}}^{-1} \bm{Hy}||^2
\end{align}
となる。

\end{document}
