\documentclass[class=jsarticle, crop=false, dvipdfmx, fleqn]{standalone}
%% preamble for Numerical-structure-analysis report

\input{/Users/User/Documents/Project/TeX/preamble/mypreamble}

%% titles
\title{先端データ解析論 レポート}
\author{37-196360 \quad 森田涼介}


%% setting for listings
\newtcbinputlisting[auto counter]{\reportlisting}[3][]{%
	listing file = {#3},
	listing options = {language=python, style=tcblatex, numbers=left, numberstyle=\tiny},
	listing only,
	breakable,
	toprule at break = 0mm,
	bottomrule at break = 0mm,
	left = 6mm,
	sharp corners,
	drop shadow,
	title = Listings \thetcbcounter : \texttt{#2},
	label = #1,
	}



%% title format
\usepackage{titlesec}
\titleformat{\section}{\LARGE}{宿題\thesection}{0zw}{}
\newcommand{\sectionbreak}{\clearpage}
\titleformat{\subsection}{\Large}{\Alph{subsection})}{0zw}{}

\begin{document}
\section{}

\(B_{\tau} (y)\)を再帰的に表現する。
\begin{align}
    B_{\tau} (y)
        & =
            \sum_{ y^{(\tau + 1)}, \ldots , y^\mathrm{(m)} = 1 }^{c}
            \exp(
                \sum_{k=\tau + 2}^{m} \bm{\zeta}^\mathrm{T} \bm{\varphi}_i^{(k)} \qty(y^{(k)},\ y^{(k-1)})
                + \bm{\zeta}^\mathrm{T} \bm{\varphi}_i^{(\tau + 1)} \qty(y^{(\tau + 1)},\ y)
                ) \\
        & =
            \sum_{y^(\tau + 1) = 1}^{c} \cdot \sum_{ y^{(\tau + 2)}, \ldots , y^\mathrm{(m)} = 1 }^{c}
            \exp{\sum_{k=\tau + 2}^{m} \bm{\zeta}^\mathrm{T} \bm{\varphi}_i^{(k)} \qty(y^{(k)},\ y^{(k-1)})}
            \exp(\bm{\zeta}^\mathrm{T} \bm{\varphi}_i^{(\tau + 1)} \qty(y^{(\tau + 1)},\ y))
\end{align}
ここで,
\begin{align}
    & \sum_{ y^{(\tau + 2)}, \ldots , y^\mathrm{(m)} = 1 }^{c}
        \exp{\sum_{k=\tau + 2}^{m} \bm{\zeta}^\mathrm{T} \bm{\varphi}_i^{(k)} \qty(y^{(k)},\ y^{(k-1)})} \notag \\
    & \ \qquad =
        \sum_{ y^{(\tau + 2)}, \ldots , y^\mathrm{(m)} = 1 }^{c}
            \exp{
                \sum_{k=\tau + 3}^{m} \bm{\zeta}^\mathrm{T} \bm{\varphi}_i^{(k)} \qty(y^{(k)},\ y^{(k-1)})
                + \bm{\zeta}^\mathrm{T} \bm{\varphi}_i^{(\tau + 2)} \qty(y^{(\tau + 2)},\ y^{(\tau + 1)})
                } \\
    & \ \qquad =
        B_{\tau + 1}\qty(y^{(\tau + 1)})
\end{align}
となるから,結局\(B_{\tau} (y)\)は,
\begin{equation}
    B_{\tau} (y) = \sum_{y^(\tau + 1) = 1}^{c} B_{\tau + 1}\qty(y^{(\tau + 1)}) \exp(\bm{\zeta}^\mathrm{T} \bm{\varphi}_i^{(\tau + 1)} \qty(y^{(\tau + 1)},\ y))
\end{equation}
という形で再帰表現される。


\end{document}
