\documentclass[dvipdfmx, fleqn, titlepage]{jsarticle}
%% preamble for Numerical-structure-analysis report

\input{/Users/User/Documents/Project/TeX/preamble/mypreamble}

%% titles
\title{先端データ解析論 レポート}
\author{37-196360 \quad 森田涼介}


%% setting for listings
\newtcbinputlisting[auto counter]{\reportlisting}[3][]{%
	listing file = {#3},
	listing options = {language=python, style=tcblatex, numbers=left, numberstyle=\tiny},
	listing only,
	breakable,
	toprule at break = 0mm,
	bottomrule at break = 0mm,
	left = 6mm,
	sharp corners,
	drop shadow,
	title = Listings \thetcbcounter : \texttt{#2},
	label = #1,
	}



%% title format
\usepackage{titlesec}
\titleformat{\section}{\LARGE}{宿題\thesection}{0zw}{}
\newcommand{\sectionbreak}{\clearpage}
\titleformat{\subsection}{\Large}{\Alph{subsection})}{0zw}{}

\title{
	先端データ解析論 \\
	第1回 レポート
	}
\begin{document}
\maketitle

\section{}

$X$を確率と統計の授業が好きであるという事象とし,
$X = 1$が好き,$X = 0$が嫌いとする。
また,$Y$を授業中に眠たいという事象とし,
$Y = 1$が眠たい,$Y = 0$が眠たくないとする。
このとき,与えられた条件より,
\begin{align*}
    & p(X=1) = 0.8 \quad p(X=0) = 0.2 \\
    & p(Y=1\ |\ X=1) = 0.25 \quad p(Y=1\ |\ X=0) = 0.25
\end{align*}
となる。


\subsection{}
\begin{align*}
    p(X=\text{好},\ Y=\text{眠})
    & = p(X=1,\ Y=1) \\
    & = p(Y=1\ |\ X=1) \cdot p(X=1) \\
    & = \num{0.25 x 0.8} \\
    & = 0.2
\end{align*}


\subsection{}
\begin{align*}
	p(Y=\text{眠})
	& = p(Y=1) \\
	& = p(Y=1\ |\ X=1) \cdot p(X=1) + p(Y=1\ |\ X=0) \cdot p(X=0) \\
	& = \num{0.25 x 0.8} + \num{0.25 x 0.2} \\
	& = 0.25
\end{align*}


\subsection{}
\begin{align*}
    p(X=\text{好}\ |\ Y=\text{眠})
    & = p(X=1\ |\ Y=1) \\
    & = \frac{p(X=1,\ Y=1)}{p(Y=1)} \\
    & = \frac{0.2}{0.25} \\
    & = 0.8
\end{align*}


\subsection{}
\begin{align*}
	& p(X=1,\ Y=1) = 0.2 \\
	& p(X=1) \cdot p(Y=1) = \num{0.8 x 0.25} = 0.2
\end{align*}
から,
\begin{equation*}
	p(X=1,\ Y=1) = p(X=1) \cdot p(Y=1)
\end{equation*}
よって,$X$と$Y$は独立である。



\newpage
\section{}
\subsection{}

離散型について,$\sum_{x} p(x) = 1$から,
\begin{align*}
	E(c) = \sum_{x} cp(x) = c \cdot \sum_x p(x) = c
\end{align*}

連続型について,$\int p(x) \dd x = 1$から,
\begin{align*}
	E(c) = \int c p(x) \dd x = c \cdot \int p(x) \dd x = c
\end{align*}


\subsection{}

離散型について,
\begin{align*}
	E(X+c)
	& = \sum_{x} (x + c) p(x) \\
	& = \sum_{x} x p(x) + \sum_{x} c p(x) \\
	& = E(X) + E(c) \\
	& = E(X) + c
\end{align*}

連続型について,
\begin{align*}
	E(X+c)
	& = \int (x + c) p(x) \dd x \\
	& = \int x p(x) \dd x + \int c p(x) \dd x \\
	& = E(X) + E(c) \\
	& = E(X) + c
\end{align*}


\subsection{}

離散型について,
\begin{align*}
	E(cX) = \sum_x cx p(x) = c \cdot \sum_x x p(x) = c E(X)
\end{align*}

連続型について,
\begin{align*}
	E(cX) = \int cx p(x) \dd x = c \cdot \int x p(x) \dd x = c E(X)
\end{align*}



\section{}
\subsection{}
$E(c) = c$から,
\begin{align*}
	V(c) = E\qty((c - E(c))^2) = E(0) = 0
\end{align*}


\subsection{}
\begin{align*}
	V(X+c)
	& = E\qty(\qty((X+c) - E(X+c))^2) \\
	& = E\qty(\qty((X+c) - (E(X) + c))^2) \\
	& = E\qty((X - E(X))^2) \\
	& = V(X)
\end{align*}


\subsection{}
$E(cX) = cE(X)$から,
\begin{align*}
	V(cX)
	& = E\qty(\qty(cX - E(cX))^2) \\
	& = E\qty(\qty(cX - cE(X))^2) \\
	& = E\qty(c^2 \qty(X - E(X))^2) \\
	& = c^2 E\qty(\qty(X - E(X))^2) \\
	& = c^2 V(X)
\end{align*}



\section{}
\subsection{}
離散型について,
\begin{align*}
	E(X + X')
	& = \sum_x \sum_{x'} (x + x') p(x,\ x') \\
	& = \sum_x \sum_{x'} x p(x,\ x') + \sum_x \sum_{x'} x' p(x,\ x') \\
	& = \sum_x x p(x) + \sum_{x'} x' p(x') \\
	& = E(X) + E(X')
\end{align*}

連続型について,
\begin{align*}
	E(X + X')
	& = \iint (x + x') p(x,\ x') \dd x \dd x' \\
	& = \iint x p(x,\ x') \dd x \dd x' + \iint x' p(x,\ x') \dd x \dd x' \\
	& = \int x p(x) \dd x + \int x' p(x') \dd x' \\
	& = E(X) + E(X')
\end{align*}


\subsection{}
\begin{align*}
	V(X + X')
	& = E\qty( \qty((X+X') - E(X+X'))^2 ) \\
	& = E\qty( \qty((X+X') - (E(X)+E(X')))^2 ) \\
	& = E\qty( \qty((X - E(X)) + (X' - E(X')))^2 ) \\
	& = E\qty( (X - E(X))^2 + 2(X - E(X))(X' - E(X')) + (X' - E(X'))^2 ) \\
	& = E\qty((X - E(X))^2) + 2 E\qty((X - E(X))(X' - E(X'))) + E\qty((X' - E(X'))^2) \\
	& = V(X) + V(X') + 2\mathrm{Cov}(X,\ X')
\end{align*}


\end{document}
