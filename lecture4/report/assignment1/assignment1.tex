\documentclass[class=jsarticle, crop=false, dvipdfmx, fleqn]{standalone}
%% preamble for Numerical-structure-analysis report

\input{/Users/User/Documents/Project/TeX/preamble/mypreamble}

%% titles
\title{先端データ解析論 レポート}
\author{37-196360 \quad 森田涼介}


%% setting for listings
\newtcbinputlisting[auto counter]{\reportlisting}[3][]{%
	listing file = {#3},
	listing options = {language=python, style=tcblatex, numbers=left, numberstyle=\tiny},
	listing only,
	breakable,
	toprule at break = 0mm,
	bottomrule at break = 0mm,
	left = 6mm,
	sharp corners,
	drop shadow,
	title = Listings \thetcbcounter : \texttt{#2},
	label = #1,
	}



%% title format
\usepackage{titlesec}
\titleformat{\section}{\LARGE}{宿題\thesection}{0zw}{}
\newcommand{\sectionbreak}{\clearpage}
\titleformat{\subsection}{\Large}{\Alph{subsection})}{0zw}{}

\begin{document}
\section{}

線形モデル
\begin{equation}
	f_{\bm{\theta}}(\bm{x})
		= \sum_{j=1}^{b} \theta_j \phi_j (\bm{x})
		= \bm{\phi}(\bm{x})^\mathrm{T} \bm{\theta}
\end{equation}
に対して,重み付き最小二乗法を考える。
ここで,
\begin{align}
	& \bm{\theta} =
		\begin{bmatrix}
			\theta_1 & \cdots & \theta_b
		\end{bmatrix}^\mathrm{T} \\
	& \bm{\phi}(\bm{x}) =
		\begin{bmatrix}
			\phi_1 (\bm{x}) & \cdots & \phi_b (\bm{x})
		\end{bmatrix}^\mathrm{T}
\end{align}
である。
このとき,損失関数は以下のように表される。
\begin{align}
	J(\bm{\theta})
		& = \frac{1}{2} \sum_{i=1}^{n} \tilde{w}_i \qty(f_{\bm{\theta}}(\bm{x}_i) - y_i)^2 \\
		& = \frac{1}{2} \sum_{i=1}^{n} \tilde{w}_i \qty(\bm{\phi}(\bm{x}_i)^\mathrm{T} \bm{\theta} - y_i)^2 \\
		& = \frac{1}{2} \sum_{i=1}^{n} \qty(\bm{\phi}(\bm{x}_i)^\mathrm{T} \bm{\theta} - y_i) \tilde{w}_i \qty(\bm{\phi}(\bm{x}_i)^\mathrm{T} \bm{\theta} - y_i) \\
		& = \frac{1}{2} (\bm{\Phi}\bm{\theta} - \bm{y})^\mathrm{T} \tilde{\bm{W}} (\bm{\Phi}\bm{\theta} - \bm{y}) \\
		& = \frac{1}{2} (\bm{\theta}^\mathrm{T} \bm{\Phi}^\mathrm{T} - \bm{y}^\mathrm{T}) \tilde{\bm{W}} (\bm{\Phi}\bm{\theta} - \bm{y}) \\
		& = \frac{1}{2} \qty{\bm{\theta}^\mathrm{T} \bm{\Phi}^\mathrm{T}\tilde{\bm{W}}\bm{\Phi}\bm{\theta} - \bm{\theta}^\mathrm{T} \bm{\Phi}^\mathrm{T} \tilde{\bm{W}}\bm{y} - \bm{y}^\mathrm{T}\tilde{\bm{W}}\bm{\Phi}\bm{\theta} + \bm{y}^\mathrm{T} \tilde{\bm{W}} \bm{y}}
\end{align}
ここで,
\begin{align}
	& \bm{\Phi} =
		\begin{bmatrix}
			\phi_1 (\bm{x}_1) & \cdots & \phi_b (\bm{x}_1) \\
			\vdots & \ddots & \vdots \\
			\phi_1 (\bm{x}_n) & \cdots & \phi_b (\bm{x}_n) \\
		\end{bmatrix} \\
	& \bm{y} =
		\begin{bmatrix}
			y_1 & \cdots & y_n
		\end{bmatrix}^\mathrm{T} \\
	& \tilde{\bm{W}} = \mathrm{diag}\qty{\tilde{w}_1,\ \cdots,\ \tilde{w}_n}
\end{align}
である。
また,\(\bm{y}^\mathrm{T}\tilde{\bm{W}}\bm{\Phi}\bm{\theta}\)はスカラーであるから転置しても値は変わらず,
\(\tilde{\bm{W}}\)は対称行列であるので転置しても\(\tilde{\bm{W}}\)のままであることに注意すると,
結局,損失関数は次のように表される。
\begin{equation}
	J(\bm{\theta}) = \frac{1}{2} \bm{\theta}^\mathrm{T} \bm{\Phi}^\mathrm{T}\tilde{\bm{W}}\bm{\Phi}\bm{\theta} - \bm{\theta}^\mathrm{T} \bm{\Phi}^\mathrm{T} \tilde{\bm{W}}\bm{y} + \frac{1}{2} \bm{y}^\mathrm{T} \tilde{\bm{W}} \bm{y}
\end{equation}
これを\(\bm{\theta}\)で偏微分すると,
\begin{equation}
	\pdv{J(\bm{\theta})}{\bm{\theta}} = \bm{\Phi}^\mathrm{T}\tilde{\bm{W}}\bm{\Phi}\bm{\theta} - \bm{\Phi}^\mathrm{T} \tilde{\bm{W}}\bm{y}
\end{equation}
となる。
これを\(\bm{0}\)とするような\(\bm{\theta}\)が\(J\)に最小を与えるので,
\(\bm{\Phi}^\mathrm{T}\tilde{\bm{W}}\bm{\Phi}\)の逆行列が存在することを仮定すると,
\begin{equation}
	\hat{\bm{\theta}} = \qty(\bm{\Phi}^\mathrm{T}\tilde{\bm{W}}\bm{\Phi})^{-1} \bm{\Phi}^\mathrm{T} \tilde{\bm{W}}\bm{y}
\end{equation}
となる。


\end{document}
