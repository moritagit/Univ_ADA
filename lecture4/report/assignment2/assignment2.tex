\documentclass[class=jsarticle, crop=false, dvipdfmx, fleqn]{standalone}
%% preamble for Numerical-structure-analysis report

\input{/Users/User/Documents/Project/TeX/preamble/mypreamble}

%% titles
\title{先端データ解析論 レポート}
\author{37-196360 \quad 森田涼介}


%% setting for listings
\newtcbinputlisting[auto counter]{\reportlisting}[3][]{%
	listing file = {#3},
	listing options = {language=python, style=tcblatex, numbers=left, numberstyle=\tiny},
	listing only,
	breakable,
	toprule at break = 0mm,
	bottomrule at break = 0mm,
	left = 6mm,
	sharp corners,
	drop shadow,
	title = Listings \thetcbcounter : \texttt{#2},
	label = #1,
	}



%% title format
\usepackage{titlesec}
\titleformat{\section}{\LARGE}{宿題\thesection}{0zw}{}
\newcommand{\sectionbreak}{\clearpage}
\titleformat{\subsection}{\Large}{\Alph{subsection})}{0zw}{}

\begin{document}
\section{}

微分可能で対称な損失\(\rho(r)\)に対して\(\tilde{r}\)で接する二次上界は,
(存在するなら)次式で与えられることを示す。
\begin{equation}
    \tilde{\rho}(r) = \frac{\tilde{w}}{2} r^2 + \mathrm{Const.}
    \qquad \qty(\tilde{w} = \frac{\rho^\prime (\tilde{r})}{\tilde{r}})
\end{equation}

微分可能で対称な損失\(\rho(r)\)に対して\(\tilde{r}\)で接する二次上界を,
\begin{equation}
    \tilde{\rho}(r) = ar^2 + br + c
\end{equation}
とおく。
\(\rho(r)\)は対称より,\(b = 0\)がわかる。
このとき,\(\tilde{\rho}(r)\)は\(\rho(r)\)に\(\tilde{r}\)で接するので,
自身の値及び1階微分の値が\(\tilde{r}\)で一致する。
したがって,
\begin{align}
    & \tilde{\rho}(\tilde{r}) = a{\tilde{r}}^2 + c = \rho(\tilde{r}) \\
    & \tilde{\rho}^\prime (\tilde{r}) = 2a\tilde{r} = \rho^\prime (\tilde{r})
\end{align}
これを解くと,
\begin{align}
    & a = \frac{\rho^\prime (\tilde{r})}{2 \tilde{r}} \\
    & c = \rho(\tilde{r}) - \frac{\rho^\prime (\tilde{r})}{2} \tilde{r}
\end{align}
となるので,結局,
\begin{align}
    \tilde{\rho}(r)
        & = \frac{\rho^\prime (\tilde{r})}{2 \tilde{r}} r + \qty(\rho(\tilde{r}) - \frac{\rho^\prime (\tilde{r})}{2} \tilde{r}) \\
        & = \frac{\tilde{w}}{2} r^2 + \mathrm{Const.}
            \qquad \qty(\tilde{w} = \frac{\rho^\prime (\tilde{r})}{\tilde{r}})
\end{align}
と表せることがわかる。


\end{document}
