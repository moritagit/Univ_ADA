\documentclass[class=jsarticle, crop=false, dvipdfmx, fleqn]{standalone}
%% preamble for Numerical-structure-analysis report

\input{/Users/User/Documents/Project/TeX/preamble/mypreamble}

%% titles
\title{先端データ解析論 レポート}
\author{37-196360 \quad 森田涼介}


%% setting for listings
\newtcbinputlisting[auto counter]{\reportlisting}[3][]{%
	listing file = {#3},
	listing options = {language=python, style=tcblatex, numbers=left, numberstyle=\tiny},
	listing only,
	breakable,
	toprule at break = 0mm,
	bottomrule at break = 0mm,
	left = 6mm,
	sharp corners,
	drop shadow,
	title = Listings \thetcbcounter : \texttt{#2},
	label = #1,
	}



%% title format
\usepackage{titlesec}
\titleformat{\section}{\LARGE}{宿題\thesection}{0zw}{}
\newcommand{\sectionbreak}{\clearpage}
\titleformat{\subsection}{\Large}{\Alph{subsection})}{0zw}{}

\begin{document}
\section{}

\begin{align}
    & \bm{W} \in \mathbb{R}^{n \times n},\ W_{i, i'} = W_{i', i}
    \label{eq:W} \\
    & \bm{D} = \mathrm{diag}\qty(\sum_{i'=1}^{n} W_{i, i'})
    \label{eq:D} \\
    & \bm{L} = \bm{D} - \bm{W}
    \label{eq:L}
\end{align}
についての固有値問題
\begin{equation}
    \bm{L} \bm{\psi} = \gamma \bm{D} \bm{\psi}
    \label{eq:eigen_problem}
\end{equation}
の最小固有値\(\gamma_n = 0\)であり,
対応する固有ベクトルは\(\bm{1}\)であることを示す。

まず,\(\bm{L}\)の最小固有値が\(0\)であることを示す。
これは,全ての固有値が非負であること,
つまり\(\bm{L}\)が半正定値行列であることを示せばよい。
よって,次式を示す。
\begin{equation}
    {}^\forall \bm{\alpha} \in \mathbb{R}^{n}: 
        \bm{\alpha}^\mathrm{T} \bm{L} \bm{\alpha} \ge 0
\end{equation}
\(\bm{\alpha},\ \bm{L}\)を要素ごとに表すと,
\begin{align}
    & \bm{\alpha} =
        \begin{bmatrix}
            \alpha_1 & \cdots & \alpha_n
        \end{bmatrix}^\mathrm{T}
    \\
    & \bm{L} =
        \begin{bmatrix}
            \qty(\sum_{i'=1}^{n} W_{1, i'} - W_{1, 1}) & -W_{1, 2} & \cdots & -W_{1, n} \\
            -W_{2, 1} & \ddots & & \vdots \\
            \vdots & & \ddots & \vdots \\
            -W_{n, 1} & \cdots & -W_{n, n-1} & \qty(\sum_{i'=1}^{n} W_{n, i'} - W_{n, n})
        \end{bmatrix}
\end{align}
となるので,
\begin{align*}
    \bm{L} \bm{\alpha}
        & =
            \begin{bmatrix}
                \qty(\sum_{i'=1}^{n} W_{1, i'} - W_{1, 1}) & -W_{1, 2} & \cdots & -W_{1, n} \\
                -W_{2, 1} & \ddots & & \vdots \\
                \vdots & & \ddots & \vdots \\
                -W_{n, 1} & \cdots & -W_{n, n-1} & \qty(\sum_{i'=1}^{n} W_{n, i'} - W_{n, n})
            \end{bmatrix}
            \cdot
            \begin{bmatrix}
                \alpha_1 \\ \vdots \\ \alpha_n
            \end{bmatrix}
        \\
        & =
            \begin{bmatrix}
                \qty(\sum_{i'=1}^{n} W_{1, i'} - W_{1, 1}) \alpha_1 - W_{1, 2} \alpha_2 - \cdots - W_{1, n} \alpha_n \\
                \vdots \\
                -W_{n, 1} \alpha_1 - \cdots - W_{n, n-1} \alpha_{n-1} + \qty(\sum_{i'=1}^{n} W_{n, i'} - W_{n, n}) \alpha_n
            \end{bmatrix}
        \\
        & =
            \begin{bmatrix}
                \qty(\sum_{i'=1}^{n} W_{1, i'}) \alpha_1 - \sum_{i'=1}^{n} W_{1, i'} \alpha_{i'} \\
                \vdots \\
                \qty(\sum_{i'=1}^{n} W_{n, i'}) \alpha_n - \sum_{i'=1}^{n} W_{n, i'} \alpha_{i'}
            \end{bmatrix}
        \\
        & =
        \begin{bmatrix}
            \sum_{i'=1}^{n} W_{1, i'} (\alpha_1 - \alpha_{i'}) \\
            \vdots \\
            \sum_{i'=1}^{n} W_{n, i'} (\alpha_n - \alpha_{i'})
        \end{bmatrix}
\end{align*}
よって,
\begin{align*}
    \bm{\alpha}^\mathrm{T} \bm{L} \bm{\alpha}
        & =
            \begin{bmatrix}
                \alpha_1 & \cdots & \alpha_n
            \end{bmatrix}
            \cdot
            \begin{bmatrix}
                \sum_{i'=1}^{n} W_{1, i'} (\alpha_1 - \alpha_{i'}) \\
                \vdots \\
                \sum_{i'=1}^{n} W_{n, i'} (\alpha_n - \alpha_{i'})
            \end{bmatrix}
        \\
        & = \sum_{i=1}^{n} \alpha_i W_{i, i'} (\alpha_i - \alpha_{i'})
        \\
        & = \sum_{i, i' = 1}^{n} W_{i, i'} \qty({\alpha_i}^2 - \alpha_i \alpha_{i'})
        \\
        & = \frac{1}{2} \sum_{i, i' = 1}^{n} W_{i, i'} \qty(2 {\alpha_i}^2 - 2 \alpha_i \alpha_{i'})
        \\
        & = \frac{1}{2} \sum_{i, i' = 1}^{n} W_{i, i'} \qty({\alpha_i}^2 + {\alpha_{i'}}^2 - 2 \alpha_i \alpha_{i'})
        \\
        & = \frac{1}{2} \sum_{i, i' = 1}^{n} W_{i, i'} \qty(\alpha_i - \alpha_{i'})^2
        \\
        & \ge 0
\end{align*}
となり,\(\bm{\alpha}^\mathrm{T} \bm{L} \bm{\alpha} \ge 0\)が示された。
これより,\(\bm{L}\)の最小固有値が\(0\)であることがわかる。

次に固有値\(0\)に対応する固有ベクトルが\(\bm{1}\)であることを示す。
\begin{equation}
    \bm{L} \bm{\psi} = \gamma \bm{D} \bm{\psi} = \bm{0}
\end{equation}
から,
\begin{equation}
    \begin{bmatrix}
        \sum_{i'=1}^{n} W_{1, i'} (\psi_1 - \psi_{i'}) \\
        \vdots \\
        \sum_{i'=1}^{n} W_{n, i'} (\psi_n - \psi_{i'})
    \end{bmatrix}
    =
    \begin{bmatrix}
        0 \\
        \vdots \\
        0
    \end{bmatrix}
\end{equation}
となる。
また,
\begin{equation}
    \bm{\psi}^\mathrm{T} \bm{L} \bm{\psi} = \frac{1}{2} \sum_{i, i' = 1}^{n} W_{i, i'} \qty(\psi_i - \psi_{i'})^2 = 0
\end{equation}
である。
これらと\(W_{i, i'} \ge 0\)から,
\begin{equation}
    \psi_i = \psi_{i'} \qquad (i,\ i' = 1,\ \cdots,\ n)
\end{equation}
となる。
これより,
\begin{equation}
    \bm{\psi} =
        \begin{bmatrix}
            \psi_1 & \cdots & \psi_n
        \end{bmatrix}
        =
        \psi_1
        \begin{bmatrix}
            1 & \cdots & 1
        \end{bmatrix}
        = \psi_1 \bm{1}
\end{equation}
となる。
よって固有値\(0\)に対応する固有ベクトルが\(\bm{1}\)であることがわかる。



\end{document}
