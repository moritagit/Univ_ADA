\documentclass[class=jsarticle, crop=false, dvipdfmx, fleqn]{standalone}
%% preamble for Numerical-structure-analysis report

\input{/Users/User/Documents/Project/TeX/preamble/mypreamble}

%% titles
\title{先端データ解析論 レポート}
\author{37-196360 \quad 森田涼介}


%% setting for listings
\newtcbinputlisting[auto counter]{\reportlisting}[3][]{%
	listing file = {#3},
	listing options = {language=python, style=tcblatex, numbers=left, numberstyle=\tiny},
	listing only,
	breakable,
	toprule at break = 0mm,
	bottomrule at break = 0mm,
	left = 6mm,
	sharp corners,
	drop shadow,
	title = Listings \thetcbcounter : \texttt{#2},
	label = #1,
	}



%% title format
\usepackage{titlesec}
\titleformat{\section}{\LARGE}{宿題\thesection}{0zw}{}
\newcommand{\sectionbreak}{\clearpage}
\titleformat{\subsection}{\Large}{\Alph{subsection})}{0zw}{}

\begin{document}
\section{}

線形モデル
\begin{equation}
	f_{\bm{\theta}}(\bm{x}) = \bm{\theta}^\mathrm{T} \bm{x}
\end{equation}
を用いた最小二乗分類を考える。
いま,訓練標本の平均\(\bm{\mu}\)について次式が成立する。
\begin{equation}
	\bm{\mu} = \frac{1}{n} \sum_{i=1}^{n} \bm{x}_i = \bm{0}
\end{equation}
また,最小二乗誤差は,
\begin{align}
	L(\bm{\theta})
		& = \frac{1}{2} \sum_{i=1}^{n} \qty(f_{\bm{\theta}}(\bm{x}) - y_i)^2 \\
		& = \frac{1}{2} ||\bm{X} \bm{\theta} - \bm{y}||^2
\end{align}
と表されるので,これの\(\bm{\theta}\)による偏微分が\(\bm{0}\)になる条件から,
\begin{equation}
	(\bm{X}^\mathrm{T} \bm{X}) \hat{\bm{\theta}} = \bm{X}^\mathrm{T} \bm{y}
	\label{eq:theta_opt}
\end{equation}
が成立する。
ここで,
\begin{align}
	& \bm{X} =
		\begin{bmatrix}
			\bm{x}_1 & \cdots & \bm{x}_n
		\end{bmatrix}^\mathrm{T} \\
	& \bm{y} =
		\begin{bmatrix}
			y_1 & \cdots & y_n
		\end{bmatrix}^\mathrm{T}
\end{align}
である。

いま,正例を下付きの\(+\)で,負例を下付きの\(-\)で表すこととすると,
次が成立する。
\begin{align}
	& n = n_+ + n_- \\
	& \bm{\mu}_+ = \frac{1}{n_+} \sum_{\bm{x}_i \in D_+} x_i \\
	& \bm{\mu}_- = \frac{1}{n_-} \sum_{\bm{x}_i \in D_-} x_i \\
\end{align}
全標本の和を考えると,
\begin{align}
	& \sum_{i=1}^{n} x_i = \sum_{\bm{x}_i \in D_+} x_i + \sum_{\bm{x}_i \in D_-} x_i \\
	& n \bm{\mu} = n_+ \bm{\mu}_+ + n_- \bm{\mu}_- = \bm{0} \\
	& \bm{\mu}_- = - \frac{n_+}{n_-} \bm{\mu}_+
\end{align}
が得られる。

また,二値分類問題を考えているので,
\(y\)は正例のとき1,負例のとき\(-1\)となる。
このことから,
\begin{align}
	\bm{X}^\mathrm{T} \bm{y}
		& = \sum_{\bm{x}_i \in D_+} x_i - \sum_{\bm{x}_i \in D_-} x_i \\
		& = n_+ \bm{\mu}_+ - n_- \bm{\mu}_- \\
		& = 2 n_+ \bm{\mu}_+
	\label{eq:XTy}
\end{align}
を得る。

また,平均\(\bm{\mu} = \bm{0}\)より,
標本の共分散行列の推定値は次で表される。
\begin{equation}
	\hat{\bm{\Sigma}}
		= \frac{1}{n} \sum_{i=1}^{n} (\bm{x}_i - \bm{\mu}) (\bm{x}_i - \bm{\mu})^\mathrm{T}
		= \frac{1}{n} \sum_{i=1}^{n} \bm{x}_i \bm{x}_i^\mathrm{T}
\end{equation}
これを用いると,次が成立する。
\begin{align}
	\bm{X}^\mathrm{T} \bm{X}
		& =
			\begin{bmatrix}
				\bm{x}_1 & \cdots & \bm{x}_n
			\end{bmatrix}
			\begin{bmatrix}
				\bm{x}_1^\mathrm{T} \\ \vdots \\ \bm{x}_n^\mathrm{T}
			\end{bmatrix} \\
		& = \sum_{i=1}^{n} \bm{x}_i \bm{x}_i^\mathrm{T} \\
		& = n \hat{\bm{\Sigma}}
	\label{eq:XTX}
\end{align}

以上,式(\ref{eq:theta_opt}),(\ref{eq:XTy}),(\ref{eq:XTX})より,
\begin{align}
	& n \hat{\bm{\Sigma}} \hat{\bm{\theta}} = 2 n_+ \bm{\mu}_+ \\
	& \hat{\bm{\theta}} = \frac{2 n_+}{n} \hat{\bm{\Sigma}}^{-1} \bm{\mu}_+
	\label{eq:boundary_LS}
\end{align}
を得る。
一方,フィッシャー判別分析による境界の垂線方向は,
\begin{align}
	\hat{\bm{\Sigma}} (\bm{\mu}_+ - \bm{\mu}_-)
		& = \hat{\bm{\Sigma}} \qty(1 + \frac{n_+}{n_-}) \bm{\mu}_+ \\
		& = \frac{n}{n_-} \hat{\bm{\Sigma}} \bm{\mu}_+
	\label{eq:boundary_Fisher}
\end{align}
となる。
式(\ref{eq:boundary_LS}),(\ref{eq:boundary_Fisher})から,
最小二乗分類により得られる識別境界の垂線\(\hat{\bm{\theta}}\)は,
フィッシャー判別分析による境界の垂線\(\hat{\bm{\Sigma}} (\bm{\mu}_+ - \bm{\mu}_-)\)と同じ方向になる。


\end{document}
