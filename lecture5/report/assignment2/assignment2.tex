\documentclass[class=jsarticle, crop=false, dvipdfmx, fleqn]{standalone}
%% preamble for Numerical-structure-analysis report

\input{/Users/User/Documents/Project/TeX/preamble/mypreamble}

%% titles
\title{先端データ解析論 レポート}
\author{37-196360 \quad 森田涼介}


%% setting for listings
\newtcbinputlisting[auto counter]{\reportlisting}[3][]{%
	listing file = {#3},
	listing options = {language=python, style=tcblatex, numbers=left, numberstyle=\tiny},
	listing only,
	breakable,
	toprule at break = 0mm,
	bottomrule at break = 0mm,
	left = 6mm,
	sharp corners,
	drop shadow,
	title = Listings \thetcbcounter : \texttt{#2},
	label = #1,
	}



%% title format
\usepackage{titlesec}
\titleformat{\section}{\LARGE}{宿題\thesection}{0zw}{}
\newcommand{\sectionbreak}{\clearpage}
\titleformat{\subsection}{\Large}{\Alph{subsection})}{0zw}{}

\begin{document}
\section{}

MNISTの手書き数字分類を,ガウスカーネルモデルに対する最小二乗回帰(一対他法)を用いて行う。
訓練標本には各500文字ずつ計5,000文字,
テスト標本には各200文字ずつ計2,000文字が含まれている。

プログラムは\pageref{listing:assignment2}ページのListing \ref{listing:assignment2}に示した。
以下に結果を示す。混同行列は表\ref{tab:confusion_matrix}のようになり,
各カテゴリごとの正解率等は表\ref{tab:result}のようになった。

ここで,ガウスカーネルのバンド幅\(h\)とL2正則化項の係数\(\lambda\)は次のようにした。
\begin{align}
    & h = 1.0 \\
    & \lambda = \num{1e-4}
\end{align}

\begin{table}[H]
	\centering
	\caption{混同行列}
	\begin{tabular}{|c||cccccccccc|} \hline
			& 0 & 1 & 2 & 3 & 4 & 5 & 6 & 7 & 8 & 9 \\ \hline\hline
		0     & 198 & 0 & 1 & 1 & 0 & 0 & 0 & 0 & 0 & 0 \\
		1     & 0 & 200 & 0 & 0 & 0 & 0 & 0 & 0 & 0 & 0 \\
		2     & 0 & 0 & 193 & 1 & 0 & 0 & 0 & 2 & 3 & 1 \\
		3     & 0 & 0 & 0 & 192 & 0 & 3 & 0 & 2 & 2 & 1 \\
		4     & 0 & 5 & 0 & 0 & 184 & 1 & 3 & 0 & 0 & 7 \\
		5     & 2 & 0 & 3 & 4 & 0 & 187 & 0 & 1 & 1 & 2 \\
		6     & 1 & 0 & 2 & 0 & 0 & 2 & 195 & 0 & 0 & 0 \\
		7     & 0 & 1 & 0 & 0 & 3 & 0 & 0 & 191 & 1 & 4 \\
		8     & 3 & 2 & 1 & 4 & 0 & 3 & 0 & 0 & 185 & 2 \\
		9     & 0 & 1 & 0 & 0 & 2 & 0 & 0 & 2 & 0 & 195 \\
		\hline
	\end{tabular}
	\label{tab:confusion_matrix}
\end{table}

\begin{table}[H]
	\centering
	\caption{テストデータに対する各カテゴリごとの結果}
	\begin{tabular}{lrrr}
		Category & \# Data & \# Correct & Accuracy \\ \hline
		0 & 200 & 198 & 0.990 \\
		1 & 200 & 200 & 1.000 \\
		2 & 200 & 193 & 0.965 \\
		3 & 200 & 192 & 0.960 \\
		4 & 200 & 184 & 0.920 \\
		5 & 200 & 187 & 0.935 \\
		6 & 200 & 195 & 0.975 \\
		7 & 200 & 191 & 0.955 \\
		8 & 200 & 185 & 0.925 \\
		9 & 200 & 195 & 0.975 \\
		All & 2,000 & 1,920 & 0.960
	\end{tabular}
	\label{tab:result}
\end{table}


\end{document}
